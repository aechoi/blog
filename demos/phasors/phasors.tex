% Options for packages loaded elsewhere
% Options for packages loaded elsewhere
\PassOptionsToPackage{unicode}{hyperref}
\PassOptionsToPackage{hyphens}{url}
\PassOptionsToPackage{dvipsnames,svgnames,x11names}{xcolor}
%
\documentclass[
  letterpaper,
  DIV=11,
  numbers=noendperiod]{scrartcl}
\usepackage{xcolor}
\usepackage{amsmath,amssymb}
\setcounter{secnumdepth}{5}
\usepackage{iftex}
\ifPDFTeX
  \usepackage[T1]{fontenc}
  \usepackage[utf8]{inputenc}
  \usepackage{textcomp} % provide euro and other symbols
\else % if luatex or xetex
  \usepackage{unicode-math} % this also loads fontspec
  \defaultfontfeatures{Scale=MatchLowercase}
  \defaultfontfeatures[\rmfamily]{Ligatures=TeX,Scale=1}
\fi
\usepackage{lmodern}
\ifPDFTeX\else
  % xetex/luatex font selection
\fi
% Use upquote if available, for straight quotes in verbatim environments
\IfFileExists{upquote.sty}{\usepackage{upquote}}{}
\IfFileExists{microtype.sty}{% use microtype if available
  \usepackage[]{microtype}
  \UseMicrotypeSet[protrusion]{basicmath} % disable protrusion for tt fonts
}{}
\makeatletter
\@ifundefined{KOMAClassName}{% if non-KOMA class
  \IfFileExists{parskip.sty}{%
    \usepackage{parskip}
  }{% else
    \setlength{\parindent}{0pt}
    \setlength{\parskip}{6pt plus 2pt minus 1pt}}
}{% if KOMA class
  \KOMAoptions{parskip=half}}
\makeatother
% Make \paragraph and \subparagraph free-standing
\makeatletter
\ifx\paragraph\undefined\else
  \let\oldparagraph\paragraph
  \renewcommand{\paragraph}{
    \@ifstar
      \xxxParagraphStar
      \xxxParagraphNoStar
  }
  \newcommand{\xxxParagraphStar}[1]{\oldparagraph*{#1}\mbox{}}
  \newcommand{\xxxParagraphNoStar}[1]{\oldparagraph{#1}\mbox{}}
\fi
\ifx\subparagraph\undefined\else
  \let\oldsubparagraph\subparagraph
  \renewcommand{\subparagraph}{
    \@ifstar
      \xxxSubParagraphStar
      \xxxSubParagraphNoStar
  }
  \newcommand{\xxxSubParagraphStar}[1]{\oldsubparagraph*{#1}\mbox{}}
  \newcommand{\xxxSubParagraphNoStar}[1]{\oldsubparagraph{#1}\mbox{}}
\fi
\makeatother

\usepackage{color}
\usepackage{fancyvrb}
\newcommand{\VerbBar}{|}
\newcommand{\VERB}{\Verb[commandchars=\\\{\}]}
\DefineVerbatimEnvironment{Highlighting}{Verbatim}{commandchars=\\\{\}}
% Add ',fontsize=\small' for more characters per line
\usepackage{framed}
\definecolor{shadecolor}{RGB}{241,243,245}
\newenvironment{Shaded}{\begin{snugshade}}{\end{snugshade}}
\newcommand{\AlertTok}[1]{\textcolor[rgb]{0.68,0.00,0.00}{#1}}
\newcommand{\AnnotationTok}[1]{\textcolor[rgb]{0.37,0.37,0.37}{#1}}
\newcommand{\AttributeTok}[1]{\textcolor[rgb]{0.40,0.45,0.13}{#1}}
\newcommand{\BaseNTok}[1]{\textcolor[rgb]{0.68,0.00,0.00}{#1}}
\newcommand{\BuiltInTok}[1]{\textcolor[rgb]{0.00,0.23,0.31}{#1}}
\newcommand{\CharTok}[1]{\textcolor[rgb]{0.13,0.47,0.30}{#1}}
\newcommand{\CommentTok}[1]{\textcolor[rgb]{0.37,0.37,0.37}{#1}}
\newcommand{\CommentVarTok}[1]{\textcolor[rgb]{0.37,0.37,0.37}{\textit{#1}}}
\newcommand{\ConstantTok}[1]{\textcolor[rgb]{0.56,0.35,0.01}{#1}}
\newcommand{\ControlFlowTok}[1]{\textcolor[rgb]{0.00,0.23,0.31}{\textbf{#1}}}
\newcommand{\DataTypeTok}[1]{\textcolor[rgb]{0.68,0.00,0.00}{#1}}
\newcommand{\DecValTok}[1]{\textcolor[rgb]{0.68,0.00,0.00}{#1}}
\newcommand{\DocumentationTok}[1]{\textcolor[rgb]{0.37,0.37,0.37}{\textit{#1}}}
\newcommand{\ErrorTok}[1]{\textcolor[rgb]{0.68,0.00,0.00}{#1}}
\newcommand{\ExtensionTok}[1]{\textcolor[rgb]{0.00,0.23,0.31}{#1}}
\newcommand{\FloatTok}[1]{\textcolor[rgb]{0.68,0.00,0.00}{#1}}
\newcommand{\FunctionTok}[1]{\textcolor[rgb]{0.28,0.35,0.67}{#1}}
\newcommand{\ImportTok}[1]{\textcolor[rgb]{0.00,0.46,0.62}{#1}}
\newcommand{\InformationTok}[1]{\textcolor[rgb]{0.37,0.37,0.37}{#1}}
\newcommand{\KeywordTok}[1]{\textcolor[rgb]{0.00,0.23,0.31}{\textbf{#1}}}
\newcommand{\NormalTok}[1]{\textcolor[rgb]{0.00,0.23,0.31}{#1}}
\newcommand{\OperatorTok}[1]{\textcolor[rgb]{0.37,0.37,0.37}{#1}}
\newcommand{\OtherTok}[1]{\textcolor[rgb]{0.00,0.23,0.31}{#1}}
\newcommand{\PreprocessorTok}[1]{\textcolor[rgb]{0.68,0.00,0.00}{#1}}
\newcommand{\RegionMarkerTok}[1]{\textcolor[rgb]{0.00,0.23,0.31}{#1}}
\newcommand{\SpecialCharTok}[1]{\textcolor[rgb]{0.37,0.37,0.37}{#1}}
\newcommand{\SpecialStringTok}[1]{\textcolor[rgb]{0.13,0.47,0.30}{#1}}
\newcommand{\StringTok}[1]{\textcolor[rgb]{0.13,0.47,0.30}{#1}}
\newcommand{\VariableTok}[1]{\textcolor[rgb]{0.07,0.07,0.07}{#1}}
\newcommand{\VerbatimStringTok}[1]{\textcolor[rgb]{0.13,0.47,0.30}{#1}}
\newcommand{\WarningTok}[1]{\textcolor[rgb]{0.37,0.37,0.37}{\textit{#1}}}

\usepackage{longtable,booktabs,array}
\usepackage{calc} % for calculating minipage widths
% Correct order of tables after \paragraph or \subparagraph
\usepackage{etoolbox}
\makeatletter
\patchcmd\longtable{\par}{\if@noskipsec\mbox{}\fi\par}{}{}
\makeatother
% Allow footnotes in longtable head/foot
\IfFileExists{footnotehyper.sty}{\usepackage{footnotehyper}}{\usepackage{footnote}}
\makesavenoteenv{longtable}
\usepackage{graphicx}
\makeatletter
\newsavebox\pandoc@box
\newcommand*\pandocbounded[1]{% scales image to fit in text height/width
  \sbox\pandoc@box{#1}%
  \Gscale@div\@tempa{\textheight}{\dimexpr\ht\pandoc@box+\dp\pandoc@box\relax}%
  \Gscale@div\@tempb{\linewidth}{\wd\pandoc@box}%
  \ifdim\@tempb\p@<\@tempa\p@\let\@tempa\@tempb\fi% select the smaller of both
  \ifdim\@tempa\p@<\p@\scalebox{\@tempa}{\usebox\pandoc@box}%
  \else\usebox{\pandoc@box}%
  \fi%
}
% Set default figure placement to htbp
\def\fps@figure{htbp}
\makeatother


% definitions for citeproc citations
\NewDocumentCommand\citeproctext{}{}
\NewDocumentCommand\citeproc{mm}{%
  \begingroup\def\citeproctext{#2}\cite{#1}\endgroup}
\makeatletter
 % allow citations to break across lines
 \let\@cite@ofmt\@firstofone
 % avoid brackets around text for \cite:
 \def\@biblabel#1{}
 \def\@cite#1#2{{#1\if@tempswa , #2\fi}}
\makeatother
\newlength{\cslhangindent}
\setlength{\cslhangindent}{1.5em}
\newlength{\csllabelwidth}
\setlength{\csllabelwidth}{3em}
\newenvironment{CSLReferences}[2] % #1 hanging-indent, #2 entry-spacing
 {\begin{list}{}{%
  \setlength{\itemindent}{0pt}
  \setlength{\leftmargin}{0pt}
  \setlength{\parsep}{0pt}
  % turn on hanging indent if param 1 is 1
  \ifodd #1
   \setlength{\leftmargin}{\cslhangindent}
   \setlength{\itemindent}{-1\cslhangindent}
  \fi
  % set entry spacing
  \setlength{\itemsep}{#2\baselineskip}}}
 {\end{list}}
\usepackage{calc}
\newcommand{\CSLBlock}[1]{\hfill\break\parbox[t]{\linewidth}{\strut\ignorespaces#1\strut}}
\newcommand{\CSLLeftMargin}[1]{\parbox[t]{\csllabelwidth}{\strut#1\strut}}
\newcommand{\CSLRightInline}[1]{\parbox[t]{\linewidth - \csllabelwidth}{\strut#1\strut}}
\newcommand{\CSLIndent}[1]{\hspace{\cslhangindent}#1}



\setlength{\emergencystretch}{3em} % prevent overfull lines

\providecommand{\tightlist}{%
  \setlength{\itemsep}{0pt}\setlength{\parskip}{0pt}}



 


\KOMAoption{captions}{tableheading}
\makeatletter
\@ifpackageloaded{tcolorbox}{}{\usepackage[skins,breakable]{tcolorbox}}
\@ifpackageloaded{fontawesome5}{}{\usepackage{fontawesome5}}
\definecolor{quarto-callout-color}{HTML}{909090}
\definecolor{quarto-callout-note-color}{HTML}{0758E5}
\definecolor{quarto-callout-important-color}{HTML}{CC1914}
\definecolor{quarto-callout-warning-color}{HTML}{EB9113}
\definecolor{quarto-callout-tip-color}{HTML}{00A047}
\definecolor{quarto-callout-caution-color}{HTML}{FC5300}
\definecolor{quarto-callout-color-frame}{HTML}{acacac}
\definecolor{quarto-callout-note-color-frame}{HTML}{4582ec}
\definecolor{quarto-callout-important-color-frame}{HTML}{d9534f}
\definecolor{quarto-callout-warning-color-frame}{HTML}{f0ad4e}
\definecolor{quarto-callout-tip-color-frame}{HTML}{02b875}
\definecolor{quarto-callout-caution-color-frame}{HTML}{fd7e14}
\makeatother
\makeatletter
\@ifpackageloaded{caption}{}{\usepackage{caption}}
\AtBeginDocument{%
\ifdefined\contentsname
  \renewcommand*\contentsname{Table of contents}
\else
  \newcommand\contentsname{Table of contents}
\fi
\ifdefined\listfigurename
  \renewcommand*\listfigurename{List of Figures}
\else
  \newcommand\listfigurename{List of Figures}
\fi
\ifdefined\listtablename
  \renewcommand*\listtablename{List of Tables}
\else
  \newcommand\listtablename{List of Tables}
\fi
\ifdefined\figurename
  \renewcommand*\figurename{Figure}
\else
  \newcommand\figurename{Figure}
\fi
\ifdefined\tablename
  \renewcommand*\tablename{Table}
\else
  \newcommand\tablename{Table}
\fi
}
\@ifpackageloaded{float}{}{\usepackage{float}}
\floatstyle{ruled}
\@ifundefined{c@chapter}{\newfloat{codelisting}{h}{lop}}{\newfloat{codelisting}{h}{lop}[chapter]}
\floatname{codelisting}{Listing}
\newcommand*\listoflistings{\listof{codelisting}{List of Listings}}
\makeatother
\makeatletter
\makeatother
\makeatletter
\@ifpackageloaded{caption}{}{\usepackage{caption}}
\@ifpackageloaded{subcaption}{}{\usepackage{subcaption}}
\makeatother
\usepackage{bookmark}
\IfFileExists{xurl.sty}{\usepackage{xurl}}{} % add URL line breaks if available
\urlstyle{same}
\hypersetup{
  pdftitle={Phasors},
  colorlinks=true,
  linkcolor={blue},
  filecolor={Maroon},
  citecolor={Blue},
  urlcolor={Blue},
  pdfcreator={LaTeX via pandoc}}


\title{Phasors}
\author{}
\date{2025-08-04}
\begin{document}
\maketitle


\begin{Shaded}
\begin{Highlighting}[]
\ImportTok{import}\NormalTok{ plotly.io }\ImportTok{as}\NormalTok{ pio}
\NormalTok{pio.renderers.default }\OperatorTok{=} \StringTok{"plotly\_mimetype+notebook\_connected"}
\end{Highlighting}
\end{Shaded}

\section{Introduction}\label{introduction}

For many, phasors can feel a bit mystical. Many students may simply
memorize the rules associated with them and run calculations without
developing a more fundamental understanding. It's not uncommon for this
to be the case at both the undergraduate and graduate level.

The goal of this document is to motivate the use of phasors from the
ground up, as well as to provide visual representations of complex
numbers that make them intuitive to understand. Hopefully by the end,
they seem like an obviously good choice for modeling sinusoidal signals.

\section{Preliminaries: Complex Numbers and Euler's
Equation}\label{preliminaries-complex-numbers-and-eulers-equation}

Before talking about sinusoids or time domain signals, let's start by
coming to grips with Euler's equation. Let's start by asking the
question, what does it mean to raise a number to an imaginary power?

\[ e^i = ?\]

Exponential notation, \(e^a\), naivly seems to imply \(a\) repeated
multiplications of \(e\). However, recall that exponents have been
extended from natural numbers to whole numbers (\(e^0 = 1\)), integers
(\(e^{-a} = 1/e^a\)), rational numbers (\(e^{1/2} = \sqrt{e}\)), and
real numbers (\href{https://xkcd.com/217/}{\(e^\pi - \pi = 20\)}). None
of these hold on to the concept of repeated multiplication. Instead,
they use a different property of exponentiation to derive these
relationships (namely,
\(e^ae^b=e^{a+b} \implies \exp(a)\exp(b) = \exp(a+b)\)).

The properties of exponentials can be generalized further to accept
complex inputs. Just like how exponentiation can be generalized from
repeated multiplication to a function that relates the product of
outputs to the sum of inputs, it can be generalized again such that it
is a function that is its own derivative.

\[
e^x = \frac{\text{d}e^x}{\text{d}x}
\]

We will first derive Euler's Equation, then examine some of its
properties.

\subsection{Deriving Euler's Equation from Picard
Iterates}\label{deriving-eulers-equation-from-picard-iterates}

Say we want to define a function, \(f(x)\), that is equal to its own
derivative. \[f(x) = f'(x)\]

You may already know what the answer is, but let's try to rederive it.
We'll start by making a guess. \[f_0(x) = 1\]

This is a pretty bad guess. We know that \(f_0'(x) = 0\), but we need
\(f_0'(x) = f_0(x) = 1\).

Watch how we'll improve this iteratively (for a truly rigorous
treatment, see
\href{https://en.wikipedia.org/wiki/Picard–Lindelöf_theorem}{Picard
Iterates}).

\begin{align}
f_0(x) &= 1 \\
f'_0(x) &= 0\\
\\
f_1(x) &= 1+x \\
f_1'(x) &= 1\\
\\
f_2(x) &= 1 + x + \frac{1}{2}x^2 \\
f'_2(x) &= 1 + x\\

&\vdots \\

f(x) &= 1 + x + \frac{x^2}{2!} + \frac{x^3}{3!} + \dots = \sum_{k=0}^\infty \frac{x^k}{k!}\\
f'(x) &= 1 + x + \frac{x^2}{2!} + \frac{x^3}{3!} + \dots = \sum_{k=0}^\infty \frac{x^k}{k!}
\end{align}

We end up with a familiar equation for the Taylor expansion of \(e^x\).
Notationally, we can keep the form of \(e^x\), but semantically we
should refer to this infinite sum.

\begin{tcolorbox}[enhanced jigsaw, breakable, coltitle=black, colframe=quarto-callout-note-color-frame, opacityback=0, left=2mm, leftrule=.75mm, titlerule=0mm, bottomtitle=1mm, colbacktitle=quarto-callout-note-color!10!white, toprule=.15mm, arc=.35mm, rightrule=.15mm, title=\textcolor{quarto-callout-note-color}{\faInfo}\hspace{0.5em}{Exercise}, colback=white, opacitybacktitle=0.6, toptitle=1mm, bottomrule=.15mm]

Verify that this does in fact satisfy the product rule for exponentials
(\(e^ae^b = e^{a+b}\)). Convince yourself that every other property of
exponentials will hold.

\end{tcolorbox}

With this form, we can simply plug in \(i\) as an argument to see what
the result is. In other words:
\[e^i = 1 +i +\frac{-1}{2} + i\frac{-1}{3!} +\dots\]

The above is just some complex number. It doesn't really simplify well,
but we can try extracting some structure by evaluating \(e^{i\theta}\)
where \(\theta \in \mathbb R\).

\[e^{i\theta} = 1 + i\theta + \frac{-\theta^2}{2} + i\frac{-\theta^3}{3!} + \dots \\
= \left(1 - \frac{\theta^2}{2} + \dots \right) + i\left(\theta - \frac{\theta^3}{3!} + \dots \right)\]

If you recall your Taylor Series, you may recognize the right hand side
as the series for \(\cos{\theta}\) and \(\sin{\theta}\). Resulting in

\[e^{i\theta} = \cos{\theta} + i\sin{\theta}\]

Thus, we rederive Euler's Equation.

The complex exponential of every other base can be obtained by recalling
the property that
\(a^{i\theta} = (e^{\ln a})^{i\theta} = e^{i\theta\ln a}\)

\subsection{Properties of Euler's
Equation}\label{properties-of-eulers-equation}

Euler's equation evaluates to some complex number.

\begin{Shaded}
\begin{Highlighting}[]
\CommentTok{\# import plotly.graph\_objects as go}
\CommentTok{\# import numpy as np}

\CommentTok{\# slider\_steps = np.linspace({-}2*np.pi, 2*np.pi, 101)}

\CommentTok{\# fig = go.Figure(}
\CommentTok{\#     data=[go.Scatter(x=[0, np.cos(0)], y=[0, np.sin(0)], line=dict(width=6),),}
\CommentTok{\#           go.Scatter(x=)],}
\CommentTok{\#     layout=go.Layout(}
\CommentTok{\#         xaxis=dict(range=[{-}2, 2], scaleanchor="y", constrain="domain"),}
\CommentTok{\#         yaxis=dict(range=[{-}2, 2]),}
\CommentTok{\#         sliders=[dict(}
\CommentTok{\#             steps=[}
\CommentTok{\#                 dict(method="animate",}
\CommentTok{\#                      args=[[f"\{i\}"],}
\CommentTok{\#                            \{"mode": "immediate", "frame": \{"duration": 0, "redraw": True\}, "transition": \{"duration": 0\}\}],}
\CommentTok{\#                      label=rf"Angle = \{np.round(angle, 2)\}")}
\CommentTok{\#                 for i, angle in enumerate(slider\_steps)}
\CommentTok{\#             ],}
\CommentTok{\#             active=int(len(slider\_steps)/2),}
\CommentTok{\#         )]}
\CommentTok{\#     ),}
\CommentTok{\#     frames=[}
\CommentTok{\#         go.Frame(}
\CommentTok{\#             data=[go.Scatter(x=[0, np.cos(angle)], y=[0, np.sin(angle)],}
\CommentTok{\#                              mode="lines+markers",}
\CommentTok{\#                              line=dict(width=6))],}
\CommentTok{\#             name=f"\{i\}"}
\CommentTok{\#         )}
\CommentTok{\#         for i, angle in enumerate(slider\_steps)}
\CommentTok{\#     ]}
\CommentTok{\# )}

\CommentTok{\# fig.add\_trace(go.Scatter(}
\CommentTok{\#     x=arc\_x,}
\CommentTok{\#     y=arc\_y,}
\CommentTok{\#     mode=\textquotesingle{}lines\textquotesingle{},}
\CommentTok{\#     line=dict(color=\textquotesingle{}red\textquotesingle{}, dash=\textquotesingle{}dot\textquotesingle{}),}
\CommentTok{\#     name=\textquotesingle{}Angle Arc\textquotesingle{}}
\CommentTok{\# ))}

\CommentTok{\# fig.show()}
\end{Highlighting}
\end{Shaded}

\begin{Shaded}
\begin{Highlighting}[]
\CommentTok{\# \# Parameters}
\CommentTok{\# theta = np.pi / 3  \# 60 degrees}
\CommentTok{\# arc\_radius = 0.5}
\CommentTok{\# arc\_resolution = 100  \# number of points on the arc}

\CommentTok{\# \# Compute arc points}
\CommentTok{\# arc\_angles = np.linspace(0, theta, arc\_resolution)}
\CommentTok{\# arc\_x = arc\_radius * np.cos(arc\_angles)}
\CommentTok{\# arc\_y = arc\_radius * np.sin(arc\_angles)}

\CommentTok{\# \# Main vector (phasor)}
\CommentTok{\# vector\_x = np.cos(theta)}
\CommentTok{\# vector\_y = np.sin(theta)}

\CommentTok{\# fig = go.Figure()}

\CommentTok{\# \# Draw the main vector}
\CommentTok{\# fig.add\_trace(go.Scatter(}
\CommentTok{\#     x=[0, vector\_x],}
\CommentTok{\#     y=[0, vector\_y],}
\CommentTok{\#     mode=\textquotesingle{}lines+markers\textquotesingle{},}
\CommentTok{\#     line=dict(color=\textquotesingle{}blue\textquotesingle{}, width=3),}
\CommentTok{\#     name=\textquotesingle{}Vector\textquotesingle{}}
\CommentTok{\# ))}

\CommentTok{\# \# Draw the arc}
\CommentTok{\# fig.add\_trace(go.Scatter(}
\CommentTok{\#     x=arc\_x,}
\CommentTok{\#     y=arc\_y,}
\CommentTok{\#     mode=\textquotesingle{}lines\textquotesingle{},}
\CommentTok{\#     line=dict(color=\textquotesingle{}red\textquotesingle{}, dash=\textquotesingle{}dot\textquotesingle{}),}
\CommentTok{\#     name=\textquotesingle{}Angle Arc\textquotesingle{}}
\CommentTok{\# ))}

\CommentTok{\# \# Optional: Add annotation for angle label}
\CommentTok{\# fig.add\_annotation(}
\CommentTok{\#     x=arc\_radius * np.cos(theta / 2),}
\CommentTok{\#     y=arc\_radius * np.sin(theta / 2),}
\CommentTok{\#     text=r\textquotesingle{}$\textbackslash{}theta$\textquotesingle{},}
\CommentTok{\#     showarrow=False,}
\CommentTok{\#     font=dict(size=16)}
\CommentTok{\# )}

\CommentTok{\# \# Aspect ratio and limits}
\CommentTok{\# fig.update\_layout(}
\CommentTok{\#     xaxis=dict(scaleanchor=\textquotesingle{}y\textquotesingle{}, range=[{-}1.2, 1.2]),}
\CommentTok{\#     yaxis=dict(range=[{-}1.2, 1.2]),}
\CommentTok{\#     width=500,}
\CommentTok{\#     height=500,}
\CommentTok{\#     showlegend=False}
\CommentTok{\# )}
\CommentTok{\# fig.show()}
\end{Highlighting}
\end{Shaded}

\phantomsection\label{refs}
\begin{CSLReferences}{0}{1}
\section{References}\label{references}

\end{CSLReferences}

\begin{center}\rule{0.5\linewidth}{0.5pt}\end{center}

\begin{tcolorbox}[enhanced jigsaw, breakable, coltitle=black, colframe=quarto-callout-note-color-frame, opacityback=0, left=2mm, leftrule=.75mm, titlerule=0mm, bottomtitle=1mm, colbacktitle=quarto-callout-note-color!10!white, toprule=.15mm, arc=.35mm, rightrule=.15mm, title=\textcolor{quarto-callout-note-color}{\faInfo}\hspace{0.5em}{Edit History}, colback=white, opacitybacktitle=0.6, toptitle=1mm, bottomrule=.15mm]

\begin{itemize}
\tightlist
\item
  \textbf{20YY-MM-DD}: EDIT NOTE
\end{itemize}

\end{tcolorbox}

\begin{center}\rule{0.5\linewidth}{0.5pt}\end{center}




\end{document}
